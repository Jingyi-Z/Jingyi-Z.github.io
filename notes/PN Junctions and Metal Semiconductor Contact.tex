% The course note for PN Junctions and Metal Semiconductor Contacts in Coursera
\documentclass{article}

\usepackage{graphicx} % import pictures from files
\usepackage{float} %set the position of table
\usepackage{geometry}  %set up page size
\geometry{a4paper, centering, scale = 0.8}
\usepackage[nottoc]{tocbibind} % add content, reference, etc.  into the content. "nottoc" means the content itself won't be added.
\usepackage{indentfirst} % indent the first line of the paragraph in each section
\usepackage{url}
\def\UrlBreaks{\do\A\do\B\do\C\do\D\do\E\do\F\do\G\do\H\do\I\do\J
	
	\do\K\do\L\do\M\do\N\do\O\do\P\do\Q\do\R\do\S\do\T\do\U\do\V
	
	\do\W\do\X\do\Y\do\Z\do\[\do\\\do\]\do\^\do\_\do\`\do\a\do\b
	
	\do\c\do\d\do\e\do\f\do\g\do\h\do\i\do\j\do\k\do\l\do\m\do\n
	
	\do\o\do\p\do\q\do\r\do\s\do\t\do\u\do\v\do\w\do\x\do\y\do\z
	
	\do\.\do\@\do\\\do\/\do\!\do\_\do\|\do\;\do\>\do\]\do\)\do\,
	
	\do\?\do\'\do+\do\=\do\#} 
\usepackage[justification=centering]{caption}



\title{\textbf{Diode - PN Junctions and Metal Semiconductor Contacts}}   %Title
\author{\texttt{J. Z.}}  %Author
\date{\texttt{Sep.4 2019}}  %Date

\bibliographystyle{plain}  %Type of bibliography
\newtheorem{thm}{Theorm} %Define a theorm environment
\newtheorem{thm2}{Definition}
\newenvironment{myquote}  
{\begin{quote} \ttfamily\itshape}
	{\end{quote}}
%set uniform format for quotations

\begin{document}  %Start of the document
	
	\maketitle  %Print the title
	\tableofcontents %Print the Content
	
	\section{PN Junction}
	
	\begin{thm2}[PN Junction]
	An \textbf{intimate contact} (interface between two materials that allows free carriers to move across the boundary) between \textbf{n-type} (cathode) and \textbf{p-type} (anode) \textbf{semiconductors}. The \textbf{forward bias} is defined when p-type is at higher voltage than n-type.
	\end{thm2}
	
	\begin{figure}[H]
		\centering
		\includegraphics[scale=0.8]{pnjunction.png}
		\caption{Crosssection of a PN junction. $https://ecee.colorado.edu/~bart/book/book/chapter4/ch4\_2.htm$ }
		\label{Fig1}
 	\end{figure}
 
    \begin{center}
    
    \end{center}
 
    Three steps to form a PN junction:  (1) Two separate p- and n-type semiconductors of the same material. (2) Two semiconductors brought into contact immediately. (3) After enough time they reach thermal equilbrium.
    
    When the device reaches equlibrium. the Fermi level is constant through the system. Far away from the PN junction, the electrical properties will be the same as isolated semiconductors.
    
    \begin{thm2}[Energy barrier]
    	The energy difference between p-type side and n-type side in the band picture.It is represented by the built-in potential $\phi_i$ and equal to the difference in Fermi energy before the junction is formed.
    \end{thm2}
    
    According to the defition of potential, 
    \[\phi_n =\frac{E_F-E_i}{q} = kT\ln\frac{n}{n_i}= kT\ln\frac{N_D}{n_i}\]
    \[\phi_p =\frac{E_i-E_F}{q} = kT\ln\frac{n_i}{p}= - kT\ln\frac{N_A}{n_i}\]
    \[\Rightarrow \phi_i = \phi_n - \phi_p = kT\ln\frac{N_DN_A}{n_i^2} and q\phi_i =(E_F)_n-(E_F)_p\]
    
   \begin{thm2}[Depletion Approximation]
    The semiconductor can be devided into two regions: the depletion region (near pn-junction, no mobile carrier) and quasi-neutral region (far from pn junction, $n = N_D\ (n-type)\ or\ N_A\ (p-type)$). At the boundry of depletion region and quasi-neutral region the carrier concentration changes \textbf{abruptly}.
   \end{thm2}
   
    
    	\subsection{PN Junction with Various Doping Profiles}
    	
    	\subsubsection{Step Junction}
    		
    		 
    		Step junction: Abrupt junction between uniformly doped p and n semiconductors.
    		Applying the depletion approximation, the Poisson's equation within the depletion region ($p=n=0$):
    		
    		\[ \frac{d^2\phi}{dx^2} = -\frac{\rho(x)}{\epsilon_s} =-\frac{q}{\epsilon_s}(p-n+N_D-N_A) = -\frac{q}{\epsilon_s}(N_D-N_A)  \]
    		
    		In n-type region($ 0<x<x_n $):
    		
    		\[
    		\frac{d^2\phi}{dx^2} =-\frac{qN_D}{\epsilon_s}  = -\frac{dE}{dx}
    		\]
    		\[
    		Boundary\ condition: E(x=x_n)=0
    		\]
    		\[
    		\Rightarrow E(x) =- \frac{qN_D}{\epsilon_s} (x_n-x)
    		\]
    		
    		In p-type region ($-x_p<x<0$):
    		
    		\[
    		\frac{d^2\phi}{dx^2} =\frac{qN_A}{\epsilon_s}  = -\frac{dE}{dx}
    		\]
    		\[
    		Boundary\ condition: E(x=-x_p)=0
    		\]
    		\[
    		\Rightarrow E(x) = -\frac{qN_A}{\epsilon_s}(x_p+x)
    		\]
    		
    		At x=0, the electric field should be continuous:
    		
    		\[
    		\frac{qN_Dx_n}{\epsilon_s} =  \frac{qN_Ax_p}{\epsilon_s}
    		\Rightarrow N_Dx_n = N_Ax_p
    		\]
    		
    		which is just the charge neurality condition. It means that total negative charges generated should be equal to positive ones.
    		
    		\subsubsection{Linearly Graded PN Junction}
	
	    \subsection{C-V and I-V Characteristics}
	
    	\subsection{Junction Breakdown}
	
    	\subsection{Heterojunction}
    	
	\section{Metal-Semiconductor Contacts}
	
	    \subsection{Schottky Contact}
	    
	    \subsection{Ohmic Contact}
	    
	\section{Optoelectronic Devices}
	
	    \subsection{Light Emitting Diode (LED)}
	    
	    \subsection{Laser Diode}
	    
	    \subsection{Photodiode}
	    
	    \subsection{Solar Cell}
	    
	
\end{document}